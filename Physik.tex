% .:: Laden der LaTeX4EI Formelsammlungsvorlage
\documentclass[fs, footer]{latex4ei}
\usepackage[european]{circuitikz}

\usepackage{multirow}
\usepackage{latexnew}


% Dokumentbeginn
% ======================================================================
\begin{document}


% Aufteilung in Spalten
\vspace{-4mm}
\begin{multicols*}{4}
	\fstitle{Physik}


	\emphbox{
	\textbf{Wichtiger Hinweis}
	\\ Diese Formelsammlung ist noch in der Entwicklung und nicht prüfungstauglich ! \\ Allerdings würden wir uns über Unterstützung freuen das zu ändern. Wer Lust hat kann uns über das Kontaktformular auf www.latex4ei.de erreichen.
	}
% ===============================================================================================
\section{Physikalische Größen und Einheiten}
%Kinematik; 
%Dynamik für Punktmassen: Kräfte & Newtonsche Gesetze; 
%Arbeit, Energie, Leistung
%Stöße zwischen Punktmassen
%Dynamik des starren Körpers


%7 SI Basiseinheiten: Sekunde, Meter, Kilogramm, Ampere, Candela, Kelvin, Mol\\ Abgeleitete Einheiten
\subsection{Messgenauigkeit und Messfehler}
Systematischer Fehler: Abweichung einer Messung von ihrem Erwartungswert\\
Statistischer Fehler: Entstehung durch zufällige positive bzw. negative Abweichungen\\
Arithmetischer Mittelwert: $\ol x = \fr{1}{n}\sum_{i=1}^n x_i$\\
Standardabweichung: $s = \sigma = \sqrt{\fr{1}{n-1}\sum_{i=1}^n (x_i-\ol x)^2}$\\
%TODO Standardabweichung mit TR: $s_{\text{Rechner}} = \sqrt{\fr{\sum_{i=1}^n x_i^2(\sum_{i=1}^n)x_i}^2}{n-1}$\\
Normalverteilung/Gauß-Funktion: $g(x) = \fr{1}{\sigma \sqrt{2\pi}\exp(-\fr{(x-\ol x)^2}{2\sigma^2})}$\\
Näherungsweise gilt: 
\begin{itemize}
\item 68\% aller Messwerte haben eine Abweichung < $\pm \sigma$ vom Mittelwert.
\item 95\% aller Messwerte haben eine Abweichung < $\pm 2\sigma$ vom Mittelwert.
\item 99,8\% aller Messwerte haben eine Abweichung < $\pm 3\sigma$ vom Mittelwert.
\end{itemize}
%TODO Fehlerfortpflanzung
%TODO Additionstheoreme


\subsection{Konstanten}
$ \epsilon_0 = 8.85\cdot 10^{12} \fr{C^2}{Nm^2}$\\

\subsection{TABELLE TRIGONOMETRISCHE FUNKTIONEN}


\section{Klassische Mechanik}
%Kinematik
%Dynamik für Punktmassen: Kräfte & Newtonsche Gesetze
%Kräfte, Arbeit, Energie, Leistung
%Stöße zwischen Punktmassen
%Dynamik des starren Körpers



\subsection{Kinetik}
momentane Geschwindigkeit: $v = \dot r$\\
mittlere Geschwindigkeit: $v_m = \fr{\Delta r}{\Delta t}$\\
\subsection{Galilei Transformation}
Gilt nur für $v<<c$\\
$x' = x - ut$ und $t' = t$ mit der Geschwindigkeit $u$ des bewegten Systems $\ra \dt x = \dt{x'} + u$\\
\subsection{Schiefe Ebene}
Gewichtskraft: $F_G = mg$\\
Normalkraft: $F_N = mg\cos \alpha$\\
Hangabtriebskraft: $F_H = F_A = mg\sin \alpha$\\
Reibung: Körper steht, falls $F_{\text{Haft}} = F_{\text{Hang}}$\\
kritischer Neigungswinkel: $tan \alpha = \mu_h$\\

\subsection{Eindimensionale Bewegungen}
Mittlere Beschleunigung: $a = \dt v$\\
Gleichförmige, geradlinige Bewegung: $x(t) = v_0t+c$\\
Gleichförmig beschleunigte Bewegung: $x(t) = \fr{1}{2}a_0t^2+v_0t+x_0$\\
\subsubsection{Konstante Geschwindigkeit}
Momentane Geschwindigkeit: $v = \dt r$\\
\subsection{Zweidimensionale Bewegungen}
Unabhängige Bewegungen in den einzelnen Raumrichtungen\\
Schiefer Wurf:
Berechnung von z(x) durch Eliminieren von t: \\$x(t) = v_{0x}t \Rightarrow t = \fr{x}{v_{0x}}$\\
$z(x) = -\fr{1}{2}g(\fr{{x}}{v_{0x}})^2 + \fr{v_{0z}}{v_{0x}}x = -\fr{g}{2v^2_{x0}}x^2 + tan\theta x$\\
\subsection{Kreisbewegung}
Winkel $\phi = \fr{s}{r}$, 	Bogenlänge $s$, Radius $r$, Umlaufdauer $T$\\
Kreisfrequenz $\omega = \dt \phi =\fr{2\pi}{f}$,	Frequenz $f$\\
Krummlinige Bewegung: $\ \v a = \dt {\v v} = \v a_t + \v a_{zp}$,	Tangetialbeschl. $a_t$\\
\subsection{Pendel}
$\omega = \sqrt{\fr{g}{l}}$\\
\subsection{Stöße}
Impuls: $p = mv$, $F=\dot p$\\
\subsubsection{Inelastischer Stoß}
Massen bilden gemeinsame Masse: $v_1' = v_2' = v'$\\
\subsubsection{Elastischer Stoß}
Fall $m_1 = m_2$: $v_1' = v_2, v_2' = v_1$\\
Fall $m_1 = m_2, v_1 \neq 0, v_2 = 0$: $v_1' = 0, v_2' = v_1$\\
$\v v_{\text{1,end}} = \frac{1}{m_1+m_2}{(m_1-m_2)\v v_{\text{1,anf}} + 2m_1\v v_{\text{2,anf}}}$\\


\subsection{Kraft}
%TODO Eventuell Newtonsche Gesetze, mt =/= const VL4
Kräfte werden vektoriell addiert: $\v F_{ges} = \sum_{i=1}^n \v F_i$\\
Gravitationskraft: $F_G = -G \fr{m_1m_2}{r_{12}^2}$, mit $G = 6,67\cdot 10^{-11} \fr{Nm^2}{kg}$\\
Zentripetalkraft: $F_Z = m \fr{v^2}{r} = m\omega r$\\
Federkraft (Hooke'sches Gesetz): $\v F_F = -k\v x$\\
mittlere Kraft: $\abs{<\v F>} = \abs{\fr{\Delta p}{\Delta t}} = \abs{\fr{m(v_E - v_A)}{\Delta t}}$\\
Coulombkraft: $F = \fr{1}{4 \pi \epsilon_0}\fr{Q_1Q_2}{r^2}$\\
Reibungskräfte allgemein: $\v F_R = \mu \v F$	,z.B. Haft-, Gleit- und Rollreibung\\
Körper beginnt zu rutschen, wenn $\mu_H >= tan \theta$\\
Luftwiderstand: $\v F_W = \frac{1}{2}\rho c_WAv^2$ mit $\rho$: Luftdichte\\

%TODO Evtl subsection Scheinkräfte VL5

\subsection{Arbeit}
Generell: $W = \int_{r_1}^{r_2} F dr$ bzw. $W = Fs\cos \alpha$\\
Spannarbeit an einer Feder: $W = \fr{1}{2} k(x-x_a)^2 $\\

\subsection{Energie}
potentielle Energie: $E_{pot} = mgh$\\
kinetische Energie: $E_{kin} = \fr{1}{2}mv^2$\\
\subsection{Energieerhaltung}
Grundprinzip: $E_{vorher} = E_{nachher}$\\
Gesamte Rotationsenergie: $E_{rot} = (\sum_{i=1}^N \fr{1}{2} \Delta m_ir_{i\perp}^2\omega ^2$\\

\subsection{Leistung}
$P = \dt W = \v F\v V = \dt E$\\



%TODO Massenmittelpunkt VL8

\subsection{Galilei-Transformation}
Transformation erleichtert Bezugssystem mit konstanter Geschwindigkeit
$\rightarrow$ Berechnung im Schwerpunktsystem

\subsection{Drehungen}
Drehmoment: $\v M = \v r \times \v F$\\
Drehimpuls: $\v L = \v r \times \v p$\\
%TODO \rightarrow \v L = mr^2 \v \omega \rightarrow \v L = J \v \omega $\\
Trägheitsmoment: $J = \sum_{i=1}^n m_iR_i^2 = \int_V r_\perp^2\rho dV$\\
%TODO VL 8 S21 Tabelle Trägheitsmomente
Satz von Steiner: $J = J_S + Md^2$ Bei bel. Achse A: Summe vom $J_S$ der Rotation durch Schwerpunkt + $Md^2$ von Schwerpunkt um A\\ %TODO BESSER
$E_{kin}(\Delta m_i)=\frac{1}{2}\Delta m_iv_i^2=\frac{1}{2}\Delta m_ir_{i\perp}^2 \omega^2$\\
Gesamte Rotationsenergie: $E_{\text{rot}}=\lim_{N \rightarrow \infty} (\sum_i^N  \fr{1}{2}\Delta m_ir_{i\perp}^2 \omega^2)=\frac{1}{2} \omega ^2\int_Vr_\perp^2 dm$\\
Für ein Teichensystem: $J = \sum_{i}m_ir_{i\perp}^2 \Rightarrow E_{\text{rot}}=\fr{1}{2}J\omega^2$\\
\subsubsection{Trägheitsmomente}
Vollzylinder: $J = \fr{1}{2}h\pi \rho R^4$\\
Hohlzylinder: $J = \fr{1}{2}h\pi \rho[R^4-(R-d)^4] \approx 2 \pi h\rho R^3d$\\
Masse des Mantels: $M \approx 2 \pi R h d \rho$\\
Energieerhalt. rollender Zylinder: $E_{\text{pot}} = E_{\text{kin,translation}} + E_{\text{rotation}} \rightarrow mgh = \fr{1}{2}mv_s^2 + \fr{1}{2}J\omega ^2; s = r\alpha, v = r\omega$\\
%TODO Tabelle Trägheitsmomente Demtröder VL9 S11
%TODO evtl analogien Translation Rotation VL10

\subsection{Planetenbewegung}
1. Keplersches Gesetz: Planetenbahnen sind Ellipsen um Stern in einem der beiden Brennpunkte\\
2. Keplersches Gesetz: In gleicher Zeit wird die gleiche Flaeche an einer Bahn aufgespannt\\$\frac{dA}{dt} = \frac{1}{2} \v r \v v sin \alpha = \frac{1}{2}\v r \times \v v = fr{1}{2m}|\v L| \Rightarrow$ Drehimpuls ist zeitlich konstant\\
3. Keplersches Gesetz: $\frac{T_1^2}{T_2^2} = \frac{a_1^3}{a_2^3}$ mit T: Umlaufzeit, a: Große Halbachse\\
%TODO evtl Spezialfall Kreisbahn





\section{Wellenlehre und Optik}
%Schwingungen
%Wellen
%Geometrische Optik
%Lichtwellen
%Laser

\subsection{Schwingungen}
\subsection{Harmonische Schwingungen} $x(t) = A \cos(\omega_0t + \Phi)$\\
A = Amplitude; $\omega$ = Kreisfrequenz [rad/s]; f = Frequenz[1/s]; T = Schwingungsdauer 1/f; $\phi$ = Phasenkonstante;\\
\subsubsection{Federpendel}
$\omega ^2 = \frac{\text{Rücktreibende Kraft}}{\text{Einheitsmasse $\times$ Einheitsauslenkung}} = \frac{k}{m};	\omega = 2 \pi f \rightarrow f = \frac{1}{2\pi}\sqrt{\frac{k}{m}}$\\
Energiebilanz: $E_{\text{ges}} = E_{\text{pot}} + E_{\text{kin}} = \frac{1}{2}kx^2+\frac{1}{2}mv^2$\\
\subsubsection{Mathematisches Pendel} $F = -mg \sin \theta \approx -mg \theta$ %TODO : Bis 15$°:$ Fehler $<1%$\\
$x = l\theta ; F=-\frac{mg}{l}x$\\
Hooke'sches Gesetz: Kraft proportional zur Auslenkung\\
\subsubsection{Torsionsschwingungen}
Elastisches Rückstelldrehmoment $M = -D\theta = J\alpha$\\
mit Torsionskonstante D und $\alpha = \frac{d^2 \theta}{dt^2}$\\
$\ddot{\theta}+\frac{D}{J}\theta = 0 \Rightarrow \omega = \sqrt{\frac{D}{J}}$\\
\subsubsection{Gedämpfter harmonischer Oszillator}
Stoke'sche Reibungskraft: $F_R = -bv = -b\dot{x}$\\
Bewegungsgleichung: $\ddot{x} + 2\gamma \dot{x} + \omega_0^2x = 0$; mit 2$\gamma$ = $\frac{b}{m}$\\
Lösungsansatz mit Cosinus: $x = Ae^{-\gamma t} \cos(\omega 't)$ \\mit $\omega ' = \sqrt{\omega_0^2-\gamma^2}$\\
schwache Dämpfung: $x = Ae^{-\gamma \frac{-t}{t_L}} \cos(\omega 't); \gamma = \frac{b}{2m}; \omega_0 = \sqrt{\frac{k}{m}}$\\
aperiodischer Grenzfall: $\gamma = \omega_0 \rightarrow	\omega' = 0$\\
überkritische Dämpfung: $\omega \ll \omega_0 \rightarrow \omega' = \sqrt{\omega_0^2 - \gamma^2}$ = img. $\omega'$ wird imaginär. Das System schwingt nicht, kehrt langsam in GGP zurück\\ 
$t_L$ = mittlere Lebensdauer, Zeit auf 1/e der Amplitude\\
\\Überlagerung von Schwingungen: $x(t)=\sum_nx_n(t)=\sum_na_n\cos{\omega_nt+\delta_n}$\\
%TODO evtl Log Dekrement VL 11, mehr zu Diffgl und erzw Schwing

\subsection{Wellen}
Polarisation in Materie: $\v P = \chi_e\varepsilon_0\v E$, mit $\chi_e:$ Elektrische Suszeptibilität, Materialeigenschaft, allg komplex\\
Longitudinale Welle: Auslenkung in Ausbreitungsrichtung\\
Transversale Welle: Auslenkung normal zur Ausbreitungsrichtung\\
Geschwindigkeit Seilwelle: $\nu = \sqrt{\frac{F_T}{\mu}}$\\ 
$F_T$ = Zugspannung, $\mu$ spezifische Masse\\ %TODO evtl S11 VL11 mit Bild
%TODO Formeln Geschwindigkeit VL11
%TODO VL12 S12 oben
Masse $m = \mu\cdot vt \rightarrow \mu = \frac{m}{vt}$\\
Elastizitätsmodul: $E = \frac{F / A}{\Delta l / l}$\\
Kompressionsmodul: $K = \frac{-p}{\Delta V / V}$\\
Ausbreitungsgeschwindigkeit Transversaal $\nu = \sqrt{\frac{F}{\mu}}$\\
Ausbreitungsgeschwindigkeit Longitudinal $\nu_\text{l} = \sqrt{\frac{E}{\rho}}$\\
Ausbreitungsgeschwindigkeit in Gasen: $\nu_\text{l,Gas} = \sqrt{\frac{K}{\rho}}$\\
Schwingungsenergie des Teilchens: $E = \frac{1}{2}kD_M^2$\\
$k = 4\pi ^2mf^2; E = 2\pi^2mf^2D_M^2$\\
$m = \rho V = \rho A v t; \Delta E = 2\pi^2 \rho A v \Delta tf^2D_M^2$\\
Durchschnittliche Leistung: $\ol{P} = fr{\Delta E}{\Delta t} = 2\pi ^2 \rho a v f^2 D_M^2$\\
%TODO Overline oben definieren und stattdessen mean, RMS; Dach als peak definieren, etc
Intensität: $ I = \fr{\ol{P}}{A} = 2\pi^2\rho v f^2D_M^2$\\
Intensität sphärische Welle: $I = \frac{\hat{P}}{r\pi r^2}$\\
$D_M \propto \frac{1}{r}$\\
%TODO Mathematische Beschreibung der Wellenausbreitung, Wellengleichung VL12
Schallpegel $L = 10\log{\frac{I}{I_0}}$dB mit $I_0 = 10^{-12}\frac{W}{m^2}$, 1dB = 10Bel\\
%TODO evtl Schallpegeltabelle
%TODO Superpositionsprinzip VL13
\\
Reflexion bei elektrischen Leitungen: $ r = \frac{Z_L-Z_0}{Z_L+Z_0}$ mit Impedanz Z\\
%TODO subsection Interferenz VL13, Überlagerung von Wellen unterschiedlicher Frequenz

\subsection{Geometrische Optik}
$f\cdot\lambda = c$\\
$c_0=2,99792458\cdot 10^8 \frac{m}{s} = \frac{1}{\sqrt{\epsilon_0 \mu_0}}$\\
%TODO was ist Licht?
Energie Photonen: $h\cdot c$ mit Plank'schem Wirkungsquantum $h=6.626\cdot10^{-34}\frac{J}{s}$\\
Brechungsindex $n = \frac{c}{v}$\\
Dielektrizitätskonstante $\varepsilon = n^2 $\\
 Brechungsgesetz von Snellius: $\frac{\sin\theta_1}{\sin\theta_2} = \frac{v_1}{v_2}  = \frac{c/n_1}{c/n_2} = \frac{n_2}{n_1}$\\
Snellius'sche Gesetz für bestimmte Winkel: $n_1\sin\theta_1 = n_2\sin\theta_2$\\
Licht bricht immer zum Medium mit dem höheren Index hin\\
Fermatsches Prinzip:\\
Licht folgt dem Weg mit der kürzesten Laufzeit$: \frac{dt}{dx} = 0$\\
%TODO evtl Fata Morgana
Optische Wand, parallelverschiebung um $\Delta d:$\\$d = t\cdot\sin(\alpha)\cdot\left[-\frac{\cos\alpha}{\sqrt{n^2- \sin ^2(\alpha}}\right]$\\
Totalreflexion: falls $\theta>\theta_g$: $\sin(\theta_g) = \frac{n2}{n1}$\\
%TODO Besser

%TODO evtl Propagationsverluste in der Glasfaser oder was auch immer
%TODO Bild vom Prisma einfügen VL15 S20
Brechungsindex n ist frequenzabhängig $n = n(\omega)$\\
Dispersion: $v(\omega)$\\
Maxwell Relation: $n = \sqrt{\varepsilon_r} = \sqrt{1+\chi_E}$\\ %Not sure
Elektrische Suszeptibilität: $\v P = N \cdot \v p$\\% %TODO war das nicht schon oben? N im text
%TODO Bewegungsgleichung erzwungene Schwingung VL16 S7
$x_0 = \frac{eE_0}{m(\omega_0^2-\omega^2}$\\
$\omega < \omega_0$: Auslenkung in Phase,
$\omega > \omega_0$: Auslenkung gegen Phase $F_{el}$\\
Dipolmoment: $|\v p(t)| = e\cdot x(t) = \left|\frac{e^2E_0\cdot sin(\omega t)}{m(\omega_0^2-}\omega^2\right|$\\ %TODO große Betragsstriche
$\chi _e(\omega) = \frac{Ne^2}{\varepsilon_0(\omega_0^2 - \omega^2)}$\\
Sellmeier Gleichung: $n^2(\lambda) = 1 + \frac{B_1\lambda^2}{\lambda^2-C_1}\frac{B_2\lambda^2}{\lambda^2-C_2}\frac{B_3\lambda^2}{\lambda^2-C_3}$\\ mit $B_i$ und $C_i$ (i $\in$ 1-3) Sellmeier Koeffizienten, experimentell ermittelt\\
Anormale Dispersion: n steigt mit $\lambda$\\
Normale Dispersion: n fällt mit $\lambda$\\
\subsection{Abbildung}
Entweder reales Bild oder virtuelles Bild (z.B. Spiegel)\\
%TODO Spiegelarten und Strahlengang evtl mit Bildern
Strahlenkonstruktion allgemein: $\frac{1}{b}= \frac{1}{f} - \frac{1}{g}$\\
fokale Länge f = $\frac{r}{2}$; Gegenstandsweite g; Bildweite b\\
Vorzeichen korrekt wählen %TODO Tabelle VL16 S18
%TODO Linsenarten mit Bildern evtl
Abbildungsmaßstab $V = \frac{B}{G}=\frac{-b}{g}$ mit V negativ: Bild umgekehrt\\
\subsubsection{Linsen}
Linsengleichung:
Gegenstandseite: $\frac{f}{g} = \frac{B}{B+G}$
Bildseite: $\frac{f}{b}= \frac{G}{G+B}$\\
Dünne vs dicke Linsen\\ %TODO
Reziproke Brennweite = Brechkraft $\rightarrow$ Einheit Dioptrie [D]= 1dpt = $\frac{1}{m}$\\
$g>f$: Reelles Bild;
$g<f$: Virtuelles Bild\\
Berechnung Brennweite: $\frac{1}{f}=(n-1)(\frac{1}{r_1}-\frac{1}{r_2})$\\
mit n = Brechungsindex der Linse, r Radizen
\\SKIZZE HIER SKIZZE HIER
%TODO SKIZZE
\subsubsection{Auge}
Weitsichtigkeit: Bild naher Gegenstände hinter Netzhaut\\
$\rightarrow$ Korrketur durch Sammelkeslinse\\
Kurzischtigkeit: Bild weiter Gegenstände vor Netzhaut\\
$\rightarrow$Korrektur durch Zerstreuungslinse\\
Stabsichtige Auge (Astigmatismus): abnormale Hornhautverkrümmung $\rightarrow$Korrektur durch Zylinderlinsen\\
\\Sehwinkel/räumliche Auflösung des Auges: $\varepsilon_0^{min} \approx 1" \Rightarrow \Delta x_{\text{min}} = S_0\cdot\varepsilon_o^{min} \approx 70 \mu m$\\
%TODO Die Lupe, Das Fernrohr VL17
Mikroskop: $V_{\text{Mikroskop}} = \frac{(l-f_e)\cdot L_d}{d_0\cdot f_e} = \beta_{\text{Objektiv}}\cdot V_{\text{Okular}}$\\
Vergrößerung Okular: $V_{\text{Okular}} = \frac{L_d}{F_e}$\\
$L_d$ = deutliche Sehweite des Mensche, ca 250mm\\
Auflösungsgrenze bei ca 1000-facher Vergrößerung\\
%TODO Dicke Linsen S16 VL 17, Fresnel Linse

\subsection{Abbildungsfehler (Abberationen)} %TODO ausarbeiten VL17,evtl mit Bildern
ausarbeiten VL17,evtl mit Bildern
\subsubsection{Schärfefehler}
Sphärische Abberationen: Groößerer EInfallswinkel am Rand $\rightarrow$ Zerstreuungskreis (Kaustik)\\
Koma
Astigmatismus
\subsubsection{Lagefehler}
Bildfeldwölbung, Verzeichnung
\subsubsection{Farbefehler/Chromatische Abberationen}
Farblängsfehler, Farbquerfehler $\rightarrow$ Dispersion\\

\subsection{Welleneigenschaft des Lichts}
$W_{\text{Welle}} = W_{\text{el}}+W_{\text{magn}} = \frac{1}{2}\cdot\varepsilon_0\cdot E^2 + \frac{1}{2\mu_0}\cdot B^2$\\
mit $E = \frac{1}{\sqrt{\mu_0\varepsilon_0}}\cdot B = c\cdot B \rightarrow W_{\text{Welle}} = \varepsilon_0E^2 = \frac{B^2}{\mu_0}$\\
Permittivität $\varepsilon$: Gibt Durchlässigkeit eines Materials für elektrische Felder an\\
magnetische Permittivität (magn Leitfähigkeit) $\mu$: Gibt Durchlässigkeit von Materie für magnetische Felder\\
$\varepsilon = \varepsilon_r\varepsilon_0; \mu = \mu_r\mu_0$\\

Welleneigenschaften: 
Pointingvektor $\v S=\frac{1}{\mu_0}\cdot \v E \times \v B = \v E \times \v H$ \\
zeigt in Ausbreitungsrichtung, Betrag = Intensität der Strahlung\\
Intensität S = Energiedichte $\times$ Ausbreitungsgeschwindigkeit\\
$[S]  = \frac{W}{m^2}$\\
Lichtwellen sind transversale e-m-Wellen mit $\v E\perp \v B \perp\v k$, mit $\v k \parallel$ Achse\\
$\v E = \v E_0\cdot\cos(k\cdot z - \omega\cdot t - \Phi) = \v E_0\cdot \cos(\frac{2\pi}{\lambda}(z-c\cdot t)-\Phi)$\\
$\v B$ ist direkt mit $\v E$ verknüpft\\
Kohärenz = Gleiche Frequenz, Feste Phasendifferenz\\
Meiste Lichtquellen inkohärent, Ausnahme Laser\\
Bei inkohärentem Licht mittelt sich die Interferenz zu null\\
Leistung eines Dipols (max $10^{-10}$m): $P = \frac{2}{3}\cdot\frac{e^2\cdot\omega^4\cdot d^2}{4\pi\varepsilon_0\cdot c^3}$\\
mit $\omega^2\cdot d = a \equiv$ Beschleunigung bei zirkularer Frequenz $\omega$\\ 
Lebensdauer atomare Schwingung: 1ns bis 10ns\\
Kohärenzlänge (Wegstrecke in 1ns): 30cm\\
%TODO evtl Fresnel'scher Doppelspiegel, Interferometer, Entspiegelung, Newtonsche Ringe
Fabry-Perot-Interferometer: Wellenlängenauflösung: $\frac{\Delta \lambda}{\lambda} = \frac{n}{N}$\\
Huygens-Fresnel-Prinzip: jeder Raumpunkt ist Ausgangspunkt für eine neue Kugelwelle (Elementarwelle)\\

\subsubsection{Beugung am Einfachspalt}
Bedingung für Minima: $a\cdot\sin\theta = Z\cdot\lambda$, mit Z $\in 0,1,2,...$\\ %TODO besser
Bedingung für Maxima: $a\cdot\sin\theta = (Z+\frac{1}{2})\cdot\lambda$, mit Z $\in 0,1,2,...$\\%TODO
\subsubsection{Beugung am Doppelspalt}
Gangunterschied $\Delta = q\cdot\sin\alpha$\\
Konstruktive Interferenz für Richtungen mit: $\Delta = Z \cdot \omega$\\
Destruktive Interferenz für Richtungen mit: $\Delta = (Z+\frac{1}{2})\omega$\\
%TODO Strichgitter, Kreuzgittertier VL19
%TODO evtl Fresnelsche Zonenplatte
\subsection{Mikroskop}
Auflösugnsvermögen Mikroskop mit Spalt b: $\Psi_{\text{min}} = \alpha = \arcsin\frac{\lambda}{b} \approx \frac{\lambda}{b}$ (Abbé Limit)\\
für runde Linse mit Durchmesser D: $D\cdot\sin\alpha = 1,22\frac{\lambda}{D}$; $\Psi_{\text{min}} = 1,22\frac{\lambda}{D}$\\
\subsubsection{Röntgenbeugung}
Bragg-Bedingung konstruktive Interferenz: $m\lambda = 2d\sin\theta$\\
\subsubsection{Polarisation von Licht}
e-m-Welle ist transversaal, also $\v E \perp \v k$ bzw $\v B\perp\v k$\\
linear polarisiert $\rightarrow$ E-Feld steht nur in eine Richtung\\
Richtung von $\v E$ ist die Polarisationsrichtung\\
von k,E aufgespannte Ebene: Polarisatonsebene\\
Emmissionsakt eines einzelnen Atoms i.d.R. polarisiert, ungeregelte Überlagerung $\rightarrow$ unpolarisiert\\
Zwei Polarisationen:
S (Senkrecht) oder P (Parallel) zur Einfallsebene\\
Einfallsebene: $\v k$ und $n$(dach) spannen Ebene auf\\
Polarisation muss gleich sein für Interferenz\\
linear, elliptisch und zirkular möglich\\
Superposition mehrerer ist möglich\\
$I' = I\cdot\cos ^2\alpha$\\





\section{Hydromechanik}
%Flüssigkeiten und Gase:
%Dichte
%Druck
%Oberflächenspannung

%Strömende Flüssigkeiten:
%Strömungen
%Reale Flüssigkeiten: Viskosität
%Strömung einer viskosen Flüssigkeit durch  ein Rohr


Dichte $\rho = \frac{m}{v}$\\
%TODO Dichte Süßwasser $ \approx \rho (T) = \rho_{\text{max}} -7 \cdot 10^{-3}(T-4)^2 $, T in °C\\
Normalkraft $\v F_N$: $\rho = \frac{F_N}{A}$\\
Schweredruck: $p_s = \rho_{\text{Fl}}\cdot h\cdot g$\\
%TODO Formel hydraulischer Lift? VL19
Kompressibilität $\kappa = -\frac{1}{p}\cdot \frac{\Delta V}{V} \rightarrow \frac{\delta V}{V}=-\kappa\cdot \delta p$\\
Kompressionsmodul K = $\frac{1}{\kappa}$\\
Schallgeschwindigkeit in Flüssigkeit: $v_0 = \sqrt{\frac{dp}{d\rho}} = \sqrt{\frac{1}{\rho\kappa}}$\\
Gewicht pro Volumen $\gamma = \rho g$ [N/$m^3$]\\
$\gamma_{\text{Wasser}} = 998\frac{kg}{m^3}\cdot 9.807\frac{m}{s^2} = 9790 \frac{N}{m^2}$\\
%TODO evtl Tabelle Dichten und Schallgeschwindigkeiten
Auftriebskraft: $F_A = \rho_{\text{Fl}}\cdot g\cdot V_K$ mit Volumen $V_K$\\
$V_{\text{Verdrängt }} = \frac{\rho_K}{\rho_{Fl}}\cdot V_K$; Einsinken bis $m_v = m_k$\\
$\rho_k < \rho_{Fl}$: Körper schwimmt\\
$\rho_k = \rho_{Fl}$: Körper schwebt\\
$\rho_k > \rho_{Fl}$: Körper sinkt\\
\\
Oberflächenspannung $\gamma = \sigma = \frac{F}{L} = \frac{dE}{dA}$ (Beide Buchstaben üblich\\
$\sigma_{\text{Wasser}} =$ 0.073 N/m\\
Kapillarspannung: $p_{\text{kap}} = \sigma(\frac{1}{r_21}+\frac{1}{r_2})$\\
kreisrunde Kapillare: $p_{\text{kap}} = \frac{2\sigma}{r}\cos (\phi) \Leftrightarrow p_S = \rho\cdot g\cdot h_{\text{kap}}$\\
%TODO Steighöhe $h_{\text{kap}} = \frac{2\sigma}{\rho gr}\cos(\phi}$\\
%TODO Bild benetzend oder nicht

\subsection{Strömungen}
laminare Ströumng: kleine Geschwindigkeiten, große innere Reibung, geringe Reibung mit Wänden\\
turbulente Strömung: große Geschwindigkeiten, geringe innere Reibung, hohe Reibung mit Wänden\\
Kontinuitätsgleichung für inkompressible Flüssigkeit: $A_1\cdot V_1 = A_2\cdot V_2$\\
Volumenstrom $V(Punkt) = \frac{dV}{dt} = A\cdot V$ ist konstant\\
%TODO evtl mit Masse des Volumenelements
Bernoulligleichung:\\
%TODO Hydrodynamisches Paradoxon

ideale Gasgleichung: $ \fr{\rho _0}{P_0} = \fr{M}{RT}$\\
Luftdruck: TODO

Viskosität: Lineares Geschwindigkeitsprofil: $F = \eta\cdot A\cdot\frac{v}{z}$ mit z = Abstand\\
Viskosität $\eta$ ist stark Temperaturabhängig\\

Strömung einer viskosen Flüssigkeit durch ein Rohr: $v(r) = \frac{p_1-p_2}{4\cdot\eta\cdot l}\cdot(R^2-r^2)$ Geschwindigkeit steigt parabelförmig zur Mitte hin an\\
Volumenstrom $ \dot V = \frac{\pi\cdot (p_1-p_2)}{2\cdot\eta\cdot l}\cdot\lfloor R^2\int_0^R r\cdot dr - \int_0^Rr^3\cdot dr\rfloor \frac{\pi\cdot(p_1-p_2)}{2\cdot\eta\cdot l}\cdot \lfloor \frac{R^4}{2}-\frac{R^4}{4}\rfloor $\\
$\rightarrow$ Gesetz von Hagen-Poiseuille: $\dot V  = \frac{\pi\cdot(p_1-p_2)}{8\cdot\eta\cdot l}\cdot R^4$\\
Reynolds Zahl: $R_e = \frac{v\cdot\rho\cdot L}{\eta}$ mit L = Charakteristische Länge/Durchmesser des Körpers:\\ 1) für $R_e >> 1$: Newtonsches Reibungsgesetz $F = c_W \cdot A\\cdot \frac{\rho\cdot v^2}{2}$ 2) für $R_e < 1$: Stokessches Reibungsgesetz $F = b\cdot v$\\
Rein laminare Strömung bei $R_e \leq 0.1$\\





\section{Thermodynamik}

Mensch Sepp, beeil dich mit deiner Vorlesung


Vielteilchensysteme $\rightarrow$ Mittelung\\

Wärmemenge Q bei Erwärmung: $Q = C_p\cdot (T_2 - T_1)$ mit $C_p =$ Wärmekapazität in $\frac{J}{K}$\\
Gaskonstante: $R = C_{\text{p(mol)}} - C_{\text{v(mol)}}$ bzw. $\nu R = C_p - C_p$\\
spezifische Wärmekapazität $c = \frac{C}{m} = \frac{\Delta Q}{\Delta T\cdot m}$ mit Wärmezufuhr $\Delta Q$, $\Delta T$ Temperaturerhöhung, $m$ Masse des Körpers\\
Wärmekapazität bei konstantem Druck (ideales Gas): $pV = \nu RT$\\

Zustandsgleichung des idealen Gases: $\rho\cdot V  = \nu\cdot R\cdot T = N\cdot k_B\cdot T$ mit $\nu$ Anzahl der Moleküle, R = 8.314 $\frac{J}{mol\cdot K}$, N Anzahl der Gasatome, $k_B = \frac{R}{N_{\text{Av}}} = 1.381\cdot 10^{-23}$ Boltzmannkonstante\\
\\Kinetische Gastheorie $pV = Nm\langle v_z^2 \rangle$\\
%TODO evtl VL 21 S6 oben
mittlere $E_{\text{text}}$ der Teilchen eines idealen Gases $\bar{E}_{\text{kin}} = \frac{3}{2}k_B T$\\
Gesamnte Translationsenergie eines idealen Gases: $\frac{3}{2}\frac{RT}{Mol}$\\

\subsection{Hauptsätze der Thermondynamik}
TODO LISTE

0. 2 Körper im thermischen Gleichgewicht zu einem dritten $\rightarrow$ Alle stehen untereinander im Gleichgewicht\\
1. $\Delta U = \Delta Q + \Delta W \rightarrow $ Es gibt kein Perpetuum mobile erster Art - Maschine mit $>$100$ $%$ Wirkungsgrad\\
Verschiedene Möglichkeiten für Zustandsänderung:\\
a) Isobarer Prozess, $p =$ const. $\rightarrow$ im idealen Gas ist $C_p$ konstant $\Rightarrow Q_{12} = C_p \Delta T$\\
b) Isochorer Prozess: $V =$ const. $\rightarrow$ im idealen Gas ist $C_v$ konstant $\Rightarrow Q_{12} = \Delta U$\\
c) Isotherme Prozesse: $T =$ const. $\Rightarrow$ $W_{12} = -Q_{12}$ Umsetzung der Wärmezufuhr in Arbeit $\Rightarrow W_{12} = \nu RT \ln\frac{V_1}{V_2}$, Freiwerdende Wärme: $Q_{12} = -W_{12}$\\
d) Adiabatische Prozesse: $\Delta Q = 0$
%TODO Isochorer Prozess, Isothermer Prozess, adiabatischer Prozess

In differentieller Schreibweise: $\delta U = \delta W + \delta Q $\\
2.Thermische Energie ist nicht in beliebigem Maße in andere Energiearten umwandelbar
\\TODO rauslöschen




%TODO Formeln mit $\propto$

%TODO evtl VL22 S9 unten

%TODO INHALTSVERZEICHNIS: Grundlagen; Das ideale Gas; Zustandsänderungen: Hauptsätze der Thermodynamik; reversible und irreversible Prozesse, Entropie; thermodynamische Maschinen


%TODO Adiabatengleichung
\subsection{Zustandsänderungen: Hauptsätze der Thermodynamik II}
Adiabatengleichung $p\cdot V^\kappa =$ const\\

Carnotscher Kreisprozess: \\
Wirkungsgrad $\eta = \frac{\vert W\vert}{Q_{12}}$, $\eta_{\text{Carnot}} = \frac{\vert -\nu R(T_2-T_1)\cdot\ln(V_2/V_1)\vert}{\nu RT_2\cdot\ln(\frac{V_2}{V_1}} = \frac{T_2-T_1}{T_2} < 1$\\



\subsection{Zustandsänderungen, Thermodynamische Systeme}


\subsection{Reversible und irreversible Prozesse}
Reversibler Prozess: z.B. Carnot- oder Stirling- Motor\\
Irreversibler Prozess $\eta{\text{irreversibel}} < \eta{\text{Carnot}}$\\

Entropie S: $\delta S = \frac{dQ_{\text{rev}}}{T} = \Delta S = \int \frac{dQ_{\text{rev}}}{T}$\\
Für ideales Gas: $\Delta S = n\cdot C_v\cdot\ln\frac{T_2}{T_1} + n\cdot R\cdot\ln\frac{V_2}{V_1}$\\
Nernst'sches Theorem: $\lim_{T \rightarrow 0} S(T) = 0$\\ d.h. Entropie am Temperaturnullpunkt ist 0\\
\\
Das reale Gas:\\
1. Endliche Ausdehnung der Moleküle: $V_{\text{real}} = V_{\text{ideal}} + nb$ mit b= Eigenvolumen von 1 Mol\\
2. Anziehung: van-der-Waals-Kraft: $p_{\text{real}} = p_{\text{ideal}} - a(\frac{n}{V^2} \rightarrow$ Druckreduktion\\
mit Materialkonstante a, welche die vdW-Kräfte berücksichtigt\\

Modifizierte Gasgleichung: $(p+a\frac{n^2}{V^2})\cdot(V-nb) = n\cdot R\cdot T$\\
\\
Wärmeleitung: $\dot Q = \frac{dQ}{dt} = - \lambda  A \frac{dT}{dx}$\\



\section{Quantenmechanik}
%Welle-Teilchen-Dualismus
%Wellenfunktion
%Unschärferelation
%Schrödingergleichung
%Lösung der Schrödingergleichung für verschiedene Potentiale


Es ist kaum noch Zeit, wie sollen wir das in ein paar Tagen schaffen?

Wärmestrahlung: Gesetz von Stefan und Boltzmann: 
Strahlungsleistung Schwarzkörper $P_s$ eines schwarzen Körpers = $ P_S = \sigma\cdot A\cdot T^4 $\\
Stefan-Boltzmann-Konstante: $\sigma = 5.670 \cdot 10^{-8}\frac{W}{m^2\cdot K^4}$\\
effektiv abgestrahlte Leistung: $\Delta P_s = \sigma\cdot A (T_1^4-T_2^4)$\\

Für nichtideale Körper $\Delta P_s = \varepsilon\sigma\cdot A (T_1^4-T_2^4)$\\
Wien'scher Verschiebungssatz: $\lambda_{max} = \frac{2897,8\mu\cdot K}{T}$\\

\subsection{Welle-Teilchen-Dualismus}
Der Photoeffekt: Metallplatte entlädt sich durch Beleuchtung mit kurzwelligem Licht\\
%TODO VL23 S18
Energie eines Photons: $E_{ph} = h\cdot f$ mit f = Frequenz des Lichts, h = Plank'sches Wirkungsquantum = $6.626\cdot 10^{-34}Js = 4.136\cdot 10^{-15}eVs$\\
Teilchen haben Welleneigenschaften\\
de Broglie-Wellenlänge $p = \frac{h}{\lambda} \Rightarrow \lambda = \frac{h}{p}$\\
Materialwellen: $\hbar = \frac{h}{2\pi} \Rightarrow p = \hbar 2\pi \frac{1}{\lambda}$\\
aus der Wellenmechanik: $2\pi\frac{1}{\lambda} = k \rightarrow \v p = \hbar \v k$\\
Impuls des Teilchens wird mit der Wellenzahl verknüpft\\

FALLS ES IHNEN NOCH NICHT AUFGEFALLEN IST, DIESE FORMELSAMMLUNG IST NOCH ECHT SCHEIẞE\\
ALARM ALARM NICHT FINAL


\end{multicols*}
\end{document}


