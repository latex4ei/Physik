% .:: Laden der LaTeX4EI Formelsammlungsvorlage
\documentclass[fs, footer]{latex4ei}
\usepackage[european]{circuitikz}

\usepackage{multirow}
\usepackage{latexnew}


% Dokumentbeginn
% ======================================================================
\begin{document}


% Aufteilung in Spalten
\vspace{-4mm}
\begin{multicols*}{4}
	\fstitle{Physik}


	\emphbox{
	\textbf{Wichtiger Hinweis}
	\\ Diese Formelsammlung ist noch in der Entwicklung und nicht prüfungstauglich ! \\ Allerdings würden wir uns über Unterstützung freuen das zu ändern. Wer Lust hat kann uns über das Kontaktformular auf www.latex4ei.de erreichen.
	}
% ===============================================================================================
\section{Messgenauigkeit und Messfehler}
Systematischer Fehler: Abweichung einer Messung von ihrem Erwartungswert\\
Statistischer Fehler: Entstehung durch zufällige positive bzw. negative Abweichungen\\
Arithmetischer Mittelwert: $\ol x = \fr{1}{n}\sum_{i=1}^n x_i$\\
Standartabweichung: $s = \sigma = \sqrt{\fr{1}{n-1}\sum_{i=1}^n (x_i-\ol x)^2}$\\
Normalverteilung/Gauß-Funktion: $g(x) = \fr{1}{\sigma \sqrt{2\pi}\exp(-\fr{(x-\ol x)^2}{2\sigma^2})}$\\
Näherungsweise gilt: 
\begin{itemize}
\item 68\% aller Messwerte haben eine Abweichung < $\pm \sigma$ vom Mittelwert.
\item 95\% aller Messwerte haben eine Abweichung < $\pm 2\sigma$ vom Mittelwert.
\item 99,8\% aller Messwerte haben eine Abweichung < $\pm 3\sigma$ vom Mittelwert.
\end{itemize}
%TODO Fehlerfortpflanzung

\section{Konstanten}
$ \epsilon_0 = 8.85*10^{12} \fr{C^2}{Nm^2}$\\

\section{Kinetik}
momentane Geschwindigkeit: $v = \dot r$\\
mittlere Geschwindigkeit: $v_m = \fr{\Delta r}{\Delta t}$\\
\subsection{Galilei Transformation}
Gilt nur für $v<<c$\\
$x' = x - ut$ und $t' = t$ mit der Geschwindigkeit $u$ des bewegten Systems $\ra \dt x = \dt{x'} + u$\\
\subsection{Schiefe Ebene}
Gewichtskraft: $F_G = mg$\\
Normalkraft: $F_N = mg\cos \alpha$\\
Hangabtriebskraft: $F_H = F_A = mg\sin \alpha$\\
Reibung: Körper steht, falls $F_{Haft} = F_{Hang}$\\
kritischer Neigungswinkel: $tan \alpha = \mu_h$\\

\subsection{Eindimensionale Bewegungen}
Mittlere Beschleunigung: $a = \dt v$\\
Gleichförmige, geradlinige Bewegung: $x(t) = v_0t+c$\\
Gleichförmig beschleunigte Bewegung: $x(t) = \fr{1}{2}a_0t^2+v_0t+x_0$\\
\subsubsection{Konstante Geschwindigkeit}
Momentane Geschwindigkeit: $v = \dt r$\\
\subsection{Zweidimensionale Bewegungen}
Unabhängige Bewegungen in den einzelnen Raumrichtungen
Schiefer Wurf:
Berechnung von z(x) durch Eliminieren von t: \\$x(t) = v_{0x}t \Rightarrow t = \fr{x}{v_{0x}}$\\
$z(x) = -\fr{1}{2}g(\fr{{x}}{v_{0x}})^2 + \fr{v_{0z}}{v_{0x}}x = -\fr{g}{2v^2_{x0}}x^2 + tan\theta x$\\
\subsection{Kreisbewegung}
Winkel $\phi = \fr{s}{r}$, 	Bogenlänge $s$, Radius $r$, Umlaufdauer $T$\\
Kreisfrequenz $\omega = \dt \phi =\fr{2\pi}{f}$,	Frequenz $f$\\
Krummlinige Bewegung: $\ \v a = \dt {\v v} = \v a_t + \v a_{zp}$,	Tangetialbeschl. $a_t$\\
\subsection{Pendel}
$\omega = \sqrt{\fr{g}{l}}$\\

\section{Kraft}
%TODO Eventuell Newtonsche Gesetze, mt =/= const VL4
Kräfte werden vektoriell addiert: $\v F_{ges} = \sum_{i=1}^n \v F_i$\\
Gravitationskraft: $F_G = -G \fr{m_1m_2}{r_{12}^2}$, mit $G = 6,67\cdot 10^{-11} \fr{Nm^2}{kg}$\\
Zentripetalkraft: $F_Z = m \fr{v^2}{r} = m\omega r$\\
Federkraft (Hooke'sches Gesetz): $\v F_F = -k\v x$\\
mittlere Kraft: $\abs{<\v F>} = \abs{\fr{\Delta p}{\Delta t}} = \abs{\fr{m(v_E - v_A)}{\Delta t}}$\\
Coulombkraft: $F = \fr{1}{4 \pi \epsilon_0}\fr{Q_1Q_2}{r^2}$\\
Reibungskräfte allgemein: $\v F_R = \mu \v F$	,z.B. Haft-, Gleit- und Rollreibung\\
Körper beginnt zu rutschen, wenn $\mu_H >= tan \theta$\\
Luftwiderstand: $\v F_W = \frac{1}{2}\rho c_WAv^2$ $\rho$ hier Luftdichte\\

%TODO Evtl subsection Scheinkräfte VL5

\section{Arbeit}
Generell: $W = \int_{r_1}^{r_2} F dr$ bzw. $W = Fs\cos \alpha$\\
Spannarbeit an einer Feder: $W = \fr{1}{2} k(x-x_a)^2 $\\

\section{Energie}
potentielle Energie: $E_{pot} = mgh$\\
kinetische Energie: $E_{kin} = \fr{1}{2}mv^2$\\
\subsection{Energieerhaltung}
Grundprinzip: $E_{vorher} = E_{nachher}$\\
Gesamte Rotationsenergie: $E_{rot} = (\sum_{i=1}^N \fr{1}{2} \Delta m_ir_{i\perp}^2\omega ^2$\\

\section{Leistung}
$P = \dt W = \v F\v V = \dt E$\\

\section{Stöße}
Impuls: $p = mv$, $F=\dot p$\\
\subsection{Inelastischer Stoß}
Massen bilden gemeinsame Masse: $v_1' = v_2' = v'$\\
\subsection{Elastischer Stoß}
Fall $m_1 = m_2$: $v_1' = v_2, v_2' = v_1$\\
Fall $m_1 = m_2, v_1 \neq 0, v_2 = 0$: $v_1' = 0, v_2' = v_1$\\
$\v v_{1,end} = \frac{1}{m_1+m_2}{(m_1-m_2)\v v{1,anf} + 2m_1\v v_{2,anf}}$\\

%TODO Massenmittelpunkt VL8

\section{Galilei-Transformation}
Transformation erleichtert Bezugssystem mit konstanter Geschwindigkeit
-> Berechnung im Schwerpunktsystem

\section{Drehungen}
Drehmoment: $\v M = \v r \times \v F$\\
Drehimpuls: $\v L = \v r \times \v p$\\
%TODO \rightarrow \v L = mr^2 \v \omega \rightarrow \v L = J \v \omega $\\
Trägheitsmoment: $J = \sum_{i=1}^n m_iR_i^2$\\
%TODO VL 8 S21 Tabelle Trägheitsmomente
Satz von Steiner: $J = J_S + Md^2$ Bei bel. Achse A: Summe vom $J_S$ der Rotation durch Schwerpunkt + $Md^2$ von Schwerpunkt um A\\



\section{Hydromechanik}
ideale Gasgleichung: $ \fr{\rho _0}{P_0} = \fr{M}{RT}$\\
Luftdruck:



	
\end{multicols*}
\end{document}


